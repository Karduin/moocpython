    
    
    
    

    

    \hypertarget{ensembles}{%
\section{Ensembles}\label{ensembles}}

    \hypertarget{compluxe9ment---niveau-basique}{%
\subsection{Complément - niveau
basique}\label{compluxe9ment---niveau-basique}}

    Ce document résume les opérations courantes disponibles sur le type
\texttt{set}. On rappelle que le type \texttt{set} est un type
\textbf{mutable}.

    \hypertarget{cruxe9ation-en-extension}{%
\subsubsection{Création en extension}\label{cruxe9ation-en-extension}}

    On crée un ensemble avec les accolades, comme les dictionnaires, mais
sans utiliser le caractère \texttt{:}, et cela donne par exemple~:

    \begin{Verbatim}[commandchars=\\\{\},frame=single,framerule=0.3mm,rulecolor=\color{cellframecolor}]
{\color{incolor}In [{\color{incolor}1}]:} \PY{n}{heteroclite} \PY{o}{=} \PY{p}{\PYZob{}}\PY{l+s+s1}{\PYZsq{}}\PY{l+s+s1}{marc}\PY{l+s+s1}{\PYZsq{}}\PY{p}{,} \PY{l+m+mi}{12}\PY{p}{,} \PY{l+s+s1}{\PYZsq{}}\PY{l+s+s1}{pierre}\PY{l+s+s1}{\PYZsq{}}\PY{p}{,} \PY{p}{(}\PY{l+m+mi}{1}\PY{p}{,} \PY{l+m+mi}{2}\PY{p}{,} \PY{l+m+mi}{3}\PY{p}{)}\PY{p}{,} \PY{l+s+s1}{\PYZsq{}}\PY{l+s+s1}{pierre}\PY{l+s+s1}{\PYZsq{}}\PY{p}{\PYZcb{}}
        \PY{n+nb}{print}\PY{p}{(}\PY{n}{heteroclite}\PY{p}{)}
\end{Verbatim}


    \begin{Verbatim}[commandchars=\\\{\},frame=single,framerule=0.3mm,rulecolor=\color{cellframecolor}]
\{'marc', 12, 'pierre', (1, 2, 3)\}
\end{Verbatim}

    \hypertarget{cruxe9ation---la-fonction-set}{%
\subsubsection{\texorpdfstring{Création - la fonction
\texttt{set}}{Création - la fonction set}}\label{cruxe9ation---la-fonction-set}}

    Il devrait être clair à ce stade que, le nom du type étant \texttt{set},
la fonction \texttt{set} est un constructeur d'ensemble. On aurait donc
aussi bien pu faire~:

    \begin{Verbatim}[commandchars=\\\{\},frame=single,framerule=0.3mm,rulecolor=\color{cellframecolor}]
{\color{incolor}In [{\color{incolor}2}]:} \PY{n}{heteroclite2} \PY{o}{=} \PY{n+nb}{set}\PY{p}{(}\PY{p}{[}\PY{l+s+s1}{\PYZsq{}}\PY{l+s+s1}{marc}\PY{l+s+s1}{\PYZsq{}}\PY{p}{,} \PY{l+m+mi}{12}\PY{p}{,} \PY{l+s+s1}{\PYZsq{}}\PY{l+s+s1}{pierre}\PY{l+s+s1}{\PYZsq{}}\PY{p}{,} \PY{p}{(}\PY{l+m+mi}{1}\PY{p}{,} \PY{l+m+mi}{2}\PY{p}{,} \PY{l+m+mi}{3}\PY{p}{)}\PY{p}{,} \PY{l+s+s1}{\PYZsq{}}\PY{l+s+s1}{pierre}\PY{l+s+s1}{\PYZsq{}}\PY{p}{]}\PY{p}{)}
        \PY{n+nb}{print}\PY{p}{(}\PY{n}{heteroclite2}\PY{p}{)}
\end{Verbatim}


    \begin{Verbatim}[commandchars=\\\{\},frame=single,framerule=0.3mm,rulecolor=\color{cellframecolor}]
\{'marc', 12, 'pierre', (1, 2, 3)\}
\end{Verbatim}

    \hypertarget{cruxe9er-un-ensemble-vide}{%
\subsubsection{Créer un ensemble vide}\label{cruxe9er-un-ensemble-vide}}

    Il faut remarquer que l'on ne peut pas créer un ensemble vide en
extension. En effet~:

    \begin{Verbatim}[commandchars=\\\{\},frame=single,framerule=0.3mm,rulecolor=\color{cellframecolor}]
{\color{incolor}In [{\color{incolor}3}]:} \PY{n+nb}{type}\PY{p}{(}\PY{p}{\PYZob{}}\PY{p}{\PYZcb{}}\PY{p}{)}
\end{Verbatim}


\begin{Verbatim}[commandchars=\\\{\},frame=single,framerule=0.3mm,rulecolor=\color{cellframecolor}]
{\color{outcolor}Out[{\color{outcolor}3}]:} dict
\end{Verbatim}
            
    Ceci est lié à des raisons historiques, les ensembles n'ayant fait leur
apparition que tardivement dans le langage en tant que citoyen de
première classe.

    Pour créer un ensemble vide, la pratique la plus courante est celle-ci~:

    \begin{Verbatim}[commandchars=\\\{\},frame=single,framerule=0.3mm,rulecolor=\color{cellframecolor}]
{\color{incolor}In [{\color{incolor}4}]:} \PY{n}{ensemble\PYZus{}vide} \PY{o}{=} \PY{n+nb}{set}\PY{p}{(}\PY{p}{)}
        \PY{n+nb}{print}\PY{p}{(}\PY{n+nb}{type}\PY{p}{(}\PY{n}{ensemble\PYZus{}vide}\PY{p}{)}\PY{p}{)}
\end{Verbatim}


    \begin{Verbatim}[commandchars=\\\{\},frame=single,framerule=0.3mm,rulecolor=\color{cellframecolor}]
<class 'set'>
\end{Verbatim}

    Ou également, moins élégant mais que l'on trouve parfois dans du vieux
code~:

    \begin{Verbatim}[commandchars=\\\{\},frame=single,framerule=0.3mm,rulecolor=\color{cellframecolor}]
{\color{incolor}In [{\color{incolor}5}]:} \PY{n}{autre\PYZus{}ensemble\PYZus{}vide} \PY{o}{=} \PY{n+nb}{set}\PY{p}{(}\PY{p}{[}\PY{p}{]}\PY{p}{)}
        \PY{n+nb}{print}\PY{p}{(}\PY{n+nb}{type}\PY{p}{(}\PY{n}{autre\PYZus{}ensemble\PYZus{}vide}\PY{p}{)}\PY{p}{)}
\end{Verbatim}


    \begin{Verbatim}[commandchars=\\\{\},frame=single,framerule=0.3mm,rulecolor=\color{cellframecolor}]
<class 'set'>
\end{Verbatim}

    \hypertarget{un-uxe9luxe9ment-dans-un-ensemble-doit-uxeatre-globalement-immuable}{%
\subsubsection{Un élément dans un ensemble doit être globalement
immuable}\label{un-uxe9luxe9ment-dans-un-ensemble-doit-uxeatre-globalement-immuable}}

    On a vu précédemment que les clés dans un dictionnaire doivent être
globalement immuables. Pour exactement les mêmes raisons, les éléments
d'un ensemble doivent aussi être globalement immuables~:

    \begin{Shaded}
\begin{Highlighting}[frame=lines,framerule=0.6mm,rulecolor=\color{asisframecolor}]
\CommentTok{# on ne peut pas insérer un tuple qui contient une liste}
\OperatorTok{>>>}\NormalTok{ ensemble }\OperatorTok{=}\NormalTok{ \{(}\DecValTok{1}\NormalTok{, }\DecValTok{2}\NormalTok{, [}\DecValTok{3}\NormalTok{, }\DecValTok{4}\NormalTok{])\}}
\NormalTok{Traceback (most recent call last):}
\NormalTok{  File }\StringTok{"<stdin>"}\NormalTok{, line }\DecValTok{1}\NormalTok{, }\KeywordTok{in} \OperatorTok{<}\NormalTok{module}\OperatorTok{>}
\PreprocessorTok{TypeError}\NormalTok{: unhashable }\BuiltInTok{type}\NormalTok{: }\StringTok{'list'}
\end{Highlighting}
\end{Shaded}

    Le type \texttt{set} étant lui-même mutable, on ne peut pas créer un
ensemble d'ensembles~:

    \begin{Shaded}
\begin{Highlighting}[frame=lines,framerule=0.6mm,rulecolor=\color{asisframecolor}]
\OperatorTok{>>>}\NormalTok{ ensemble }\OperatorTok{=}\NormalTok{ \{\{}\DecValTok{1}\NormalTok{, }\DecValTok{2}\NormalTok{\}\}}
\NormalTok{Traceback (most recent call last):}
\NormalTok{  File }\StringTok{"<stdin>"}\NormalTok{, line }\DecValTok{1}\NormalTok{, }\KeywordTok{in} \OperatorTok{<}\NormalTok{module}\OperatorTok{>}
\PreprocessorTok{TypeError}\NormalTok{: unhashable }\BuiltInTok{type}\NormalTok{: }\StringTok{'set'}
\end{Highlighting}
\end{Shaded}

    Et c'est une des raisons d'être du type \texttt{frozenset}.

    \hypertarget{cruxe9ation---la-fonction-frozenset}{%
\subsubsection{\texorpdfstring{Création - la fonction
\texttt{frozenset}}{Création - la fonction frozenset}}\label{cruxe9ation---la-fonction-frozenset}}

    Un \texttt{frozenset} est un ensemble qu'on ne peut pas modifier, et qui
donc peut servir de clé dans un dictionnaire, ou être inclus dans un
autre ensemble (mutable ou pas).

    Il n'existe pas de raccourci syntaxique comme les \texttt{\{\}} pour
créer un ensemble immuable, qui doit être créé avec la fonction
\texttt{frozenset}. Toutes les opérations documentées dans ce notebook,
et qui n'ont pas besoin de modifier l'ensemble, sont disponibles sur un
\texttt{frozenset}.

Parmi les fonctions exclues sur un \texttt{frozenset}, on peut citer~:
\texttt{update}, \texttt{pop}, \texttt{clear}, \texttt{remove} ou
\texttt{discard}.

    \hypertarget{opuxe9rations-simples}{%
\subsubsection{Opérations simples}\label{opuxe9rations-simples}}

    \begin{Verbatim}[commandchars=\\\{\},frame=single,framerule=0.3mm,rulecolor=\color{cellframecolor}]
{\color{incolor}In [{\color{incolor}6}]:} \PY{c+c1}{\PYZsh{} pour rappel}
        \PY{n}{heteroclite}
\end{Verbatim}


\begin{Verbatim}[commandchars=\\\{\},frame=single,framerule=0.3mm,rulecolor=\color{cellframecolor}]
{\color{outcolor}Out[{\color{outcolor}6}]:} \{(1, 2, 3), 12, 'marc', 'pierre'\}
\end{Verbatim}
            
    \hypertarget{test-dappartenance}{%
\subparagraph{Test d'appartenance}\label{test-dappartenance}}

    \begin{Verbatim}[commandchars=\\\{\},frame=single,framerule=0.3mm,rulecolor=\color{cellframecolor}]
{\color{incolor}In [{\color{incolor}7}]:} \PY{p}{(}\PY{l+m+mi}{1}\PY{p}{,} \PY{l+m+mi}{2}\PY{p}{,} \PY{l+m+mi}{3}\PY{p}{)} \PY{o+ow}{in} \PY{n}{heteroclite}
\end{Verbatim}


\begin{Verbatim}[commandchars=\\\{\},frame=single,framerule=0.3mm,rulecolor=\color{cellframecolor}]
{\color{outcolor}Out[{\color{outcolor}7}]:} True
\end{Verbatim}
            
    \hypertarget{cardinal}{%
\subparagraph{Cardinal}\label{cardinal}}

    \begin{Verbatim}[commandchars=\\\{\},frame=single,framerule=0.3mm,rulecolor=\color{cellframecolor}]
{\color{incolor}In [{\color{incolor}8}]:} \PY{n+nb}{len}\PY{p}{(}\PY{n}{heteroclite}\PY{p}{)}
\end{Verbatim}


\begin{Verbatim}[commandchars=\\\{\},frame=single,framerule=0.3mm,rulecolor=\color{cellframecolor}]
{\color{outcolor}Out[{\color{outcolor}8}]:} 4
\end{Verbatim}
            
    \hypertarget{manipulations}{%
\subparagraph{Manipulations}\label{manipulations}}

    \begin{Verbatim}[commandchars=\\\{\},frame=single,framerule=0.3mm,rulecolor=\color{cellframecolor}]
{\color{incolor}In [{\color{incolor}9}]:} \PY{n}{ensemble} \PY{o}{=} \PY{p}{\PYZob{}}\PY{l+m+mi}{1}\PY{p}{,} \PY{l+m+mi}{2}\PY{p}{,} \PY{l+m+mi}{1}\PY{p}{\PYZcb{}}
        \PY{n}{ensemble}
\end{Verbatim}


\begin{Verbatim}[commandchars=\\\{\},frame=single,framerule=0.3mm,rulecolor=\color{cellframecolor}]
{\color{outcolor}Out[{\color{outcolor}9}]:} \{1, 2\}
\end{Verbatim}
            
    \begin{Verbatim}[commandchars=\\\{\},frame=single,framerule=0.3mm,rulecolor=\color{cellframecolor}]
{\color{incolor}In [{\color{incolor}10}]:} \PY{c+c1}{\PYZsh{} pour nettoyer}
         \PY{n}{ensemble}\PY{o}{.}\PY{n}{clear}\PY{p}{(}\PY{p}{)}
         \PY{n}{ensemble}
\end{Verbatim}


\begin{Verbatim}[commandchars=\\\{\},frame=single,framerule=0.3mm,rulecolor=\color{cellframecolor}]
{\color{outcolor}Out[{\color{outcolor}10}]:} set()
\end{Verbatim}
            
    \begin{Verbatim}[commandchars=\\\{\},frame=single,framerule=0.3mm,rulecolor=\color{cellframecolor}]
{\color{incolor}In [{\color{incolor}11}]:} \PY{c+c1}{\PYZsh{} ajouter un element}
         \PY{n}{ensemble}\PY{o}{.}\PY{n}{add}\PY{p}{(}\PY{l+m+mi}{1}\PY{p}{)}
         \PY{n}{ensemble}
\end{Verbatim}


\begin{Verbatim}[commandchars=\\\{\},frame=single,framerule=0.3mm,rulecolor=\color{cellframecolor}]
{\color{outcolor}Out[{\color{outcolor}11}]:} \{1\}
\end{Verbatim}
            
    \begin{Verbatim}[commandchars=\\\{\},frame=single,framerule=0.3mm,rulecolor=\color{cellframecolor}]
{\color{incolor}In [{\color{incolor}12}]:} \PY{c+c1}{\PYZsh{} ajouter tous les elements d\PYZsq{}un autre *ensemble*}
         \PY{n}{ensemble}\PY{o}{.}\PY{n}{update}\PY{p}{(}\PY{p}{\PYZob{}}\PY{l+m+mi}{2}\PY{p}{,} \PY{p}{(}\PY{l+m+mi}{1}\PY{p}{,} \PY{l+m+mi}{2}\PY{p}{,} \PY{l+m+mi}{3}\PY{p}{)}\PY{p}{,} \PY{p}{(}\PY{l+m+mi}{1}\PY{p}{,} \PY{l+m+mi}{3}\PY{p}{,} \PY{l+m+mi}{5}\PY{p}{)}\PY{p}{\PYZcb{}}\PY{p}{)}
         \PY{n}{ensemble}
\end{Verbatim}


\begin{Verbatim}[commandchars=\\\{\},frame=single,framerule=0.3mm,rulecolor=\color{cellframecolor}]
{\color{outcolor}Out[{\color{outcolor}12}]:} \{(1, 2, 3), (1, 3, 5), 1, 2\}
\end{Verbatim}
            
    \begin{Verbatim}[commandchars=\\\{\},frame=single,framerule=0.3mm,rulecolor=\color{cellframecolor}]
{\color{incolor}In [{\color{incolor}13}]:} \PY{c+c1}{\PYZsh{} enlever un element avec discard}
         \PY{n}{ensemble}\PY{o}{.}\PY{n}{discard}\PY{p}{(}\PY{p}{(}\PY{l+m+mi}{1}\PY{p}{,} \PY{l+m+mi}{3}\PY{p}{,} \PY{l+m+mi}{5}\PY{p}{)}\PY{p}{)}
         \PY{n}{ensemble}
\end{Verbatim}


\begin{Verbatim}[commandchars=\\\{\},frame=single,framerule=0.3mm,rulecolor=\color{cellframecolor}]
{\color{outcolor}Out[{\color{outcolor}13}]:} \{(1, 2, 3), 1, 2\}
\end{Verbatim}
            
    \begin{Verbatim}[commandchars=\\\{\},frame=single,framerule=0.3mm,rulecolor=\color{cellframecolor}]
{\color{incolor}In [{\color{incolor}14}]:} \PY{c+c1}{\PYZsh{} discard fonctionne même si l\PYZsq{}élément n\PYZsq{}est pas présent}
         \PY{n}{ensemble}\PY{o}{.}\PY{n}{discard}\PY{p}{(}\PY{l+s+s1}{\PYZsq{}}\PY{l+s+s1}{foo}\PY{l+s+s1}{\PYZsq{}}\PY{p}{)}
         \PY{n}{ensemble}
\end{Verbatim}


\begin{Verbatim}[commandchars=\\\{\},frame=single,framerule=0.3mm,rulecolor=\color{cellframecolor}]
{\color{outcolor}Out[{\color{outcolor}14}]:} \{(1, 2, 3), 1, 2\}
\end{Verbatim}
            
    \begin{Verbatim}[commandchars=\\\{\},frame=single,framerule=0.3mm,rulecolor=\color{cellframecolor}]
{\color{incolor}In [{\color{incolor}15}]:} \PY{c+c1}{\PYZsh{} enlever un élément avec remove}
         \PY{n}{ensemble}\PY{o}{.}\PY{n}{remove}\PY{p}{(}\PY{p}{(}\PY{l+m+mi}{1}\PY{p}{,} \PY{l+m+mi}{2}\PY{p}{,} \PY{l+m+mi}{3}\PY{p}{)}\PY{p}{)}
         \PY{n}{ensemble}
\end{Verbatim}


\begin{Verbatim}[commandchars=\\\{\},frame=single,framerule=0.3mm,rulecolor=\color{cellframecolor}]
{\color{outcolor}Out[{\color{outcolor}15}]:} \{1, 2\}
\end{Verbatim}
            
    \begin{Verbatim}[commandchars=\\\{\},frame=single,framerule=0.3mm,rulecolor=\color{cellframecolor}]
{\color{incolor}In [{\color{incolor}16}]:} \PY{c+c1}{\PYZsh{} contrairement à discard, l\PYZsq{}élément doit être présent,}
         \PY{c+c1}{\PYZsh{} sinon il y a une exception}
         \PY{k}{try}\PY{p}{:}
             \PY{n}{ensemble}\PY{o}{.}\PY{n}{remove}\PY{p}{(}\PY{l+s+s1}{\PYZsq{}}\PY{l+s+s1}{foo}\PY{l+s+s1}{\PYZsq{}}\PY{p}{)}
         \PY{k}{except} \PY{n+ne}{KeyError} \PY{k}{as} \PY{n}{e}\PY{p}{:}
             \PY{n+nb}{print}\PY{p}{(}\PY{l+s+s2}{\PYZdq{}}\PY{l+s+s2}{remove a levé l}\PY{l+s+s2}{\PYZsq{}}\PY{l+s+s2}{exception}\PY{l+s+s2}{\PYZdq{}}\PY{p}{,} \PY{n}{e}\PY{p}{)}
\end{Verbatim}


    \begin{Verbatim}[commandchars=\\\{\},frame=single,framerule=0.3mm,rulecolor=\color{cellframecolor}]
remove a levé l'exception 'foo'
\end{Verbatim}

    La capture d'exception avec \texttt{try} et \texttt{except} sert à
capturer une erreur d'exécution du programme (que l'on appelle
exception) pour continuer le programme. Le but de cet exemple est
simplement de montrer (d'une manière plus élégante que de voir
simplement le programme planter avec une exception non capturée) que
l'expression
\texttt{ensemble.remove(\textquotesingle{}foo\textquotesingle{})} génère
une exception. Si ce concept vous paraît obscur, pas d'inquiétude, nous
l'aborderons cette semaine et nous y reviendrons en détail en semaine 6.

    \begin{Verbatim}[commandchars=\\\{\},frame=single,framerule=0.3mm,rulecolor=\color{cellframecolor}]
{\color{incolor}In [{\color{incolor}17}]:} \PY{c+c1}{\PYZsh{} pop() ressemble à la méthode éponyme sur les listes}
         \PY{c+c1}{\PYZsh{} sauf qu\PYZsq{}il n\PYZsq{}y a pas d\PYZsq{}ordre dans un ensemble}
         \PY{k}{while} \PY{n}{ensemble}\PY{p}{:}
             \PY{n}{element} \PY{o}{=} \PY{n}{ensemble}\PY{o}{.}\PY{n}{pop}\PY{p}{(}\PY{p}{)}
             \PY{n+nb}{print}\PY{p}{(}\PY{l+s+s2}{\PYZdq{}}\PY{l+s+s2}{element}\PY{l+s+s2}{\PYZdq{}}\PY{p}{,} \PY{n}{element}\PY{p}{)}
         \PY{n+nb}{print}\PY{p}{(}\PY{l+s+s2}{\PYZdq{}}\PY{l+s+s2}{et bien sûr maintenant l}\PY{l+s+s2}{\PYZsq{}}\PY{l+s+s2}{ensemble est vide}\PY{l+s+s2}{\PYZdq{}}\PY{p}{,} \PY{n}{ensemble}\PY{p}{)}
\end{Verbatim}


    \begin{Verbatim}[commandchars=\\\{\},frame=single,framerule=0.3mm,rulecolor=\color{cellframecolor}]
element 1
element 2
et bien sûr maintenant l'ensemble est vide set()
\end{Verbatim}

    \hypertarget{opuxe9rations-classiques-sur-les-ensembles}{%
\subsubsection{Opérations classiques sur les
ensembles}\label{opuxe9rations-classiques-sur-les-ensembles}}

    Donnons-nous deux ensembles simples~:

    \begin{Verbatim}[commandchars=\\\{\},frame=single,framerule=0.3mm,rulecolor=\color{cellframecolor}]
{\color{incolor}In [{\color{incolor}18}]:} \PY{n}{A2} \PY{o}{=} \PY{n+nb}{set}\PY{p}{(}\PY{p}{[}\PY{l+m+mi}{0}\PY{p}{,} \PY{l+m+mi}{2}\PY{p}{,} \PY{l+m+mi}{4}\PY{p}{,} \PY{l+m+mi}{6}\PY{p}{]}\PY{p}{)}
         \PY{n+nb}{print}\PY{p}{(}\PY{l+s+s1}{\PYZsq{}}\PY{l+s+s1}{A2}\PY{l+s+s1}{\PYZsq{}}\PY{p}{,} \PY{n}{A2}\PY{p}{)}
         \PY{n}{A3} \PY{o}{=} \PY{n+nb}{set}\PY{p}{(}\PY{p}{[}\PY{l+m+mi}{0}\PY{p}{,} \PY{l+m+mi}{6}\PY{p}{,} \PY{l+m+mi}{3}\PY{p}{]}\PY{p}{)}
         \PY{n+nb}{print}\PY{p}{(}\PY{l+s+s1}{\PYZsq{}}\PY{l+s+s1}{A3}\PY{l+s+s1}{\PYZsq{}}\PY{p}{,} \PY{n}{A3}\PY{p}{)}
\end{Verbatim}


    \begin{Verbatim}[commandchars=\\\{\},frame=single,framerule=0.3mm,rulecolor=\color{cellframecolor}]
A2 \{0, 2, 4, 6\}
A3 \{0, 3, 6\}
\end{Verbatim}

    N'oubliez pas que les ensembles, comme les dictionnaires, ne sont
\textbf{pas ordonnés}.

    \textbf{Remarques}~:

\begin{itemize}
\tightlist
\item
  les notations des opérateurs sur les ensembles rappellent les
  opérateurs ``bit-à-bit'' sur les entiers~;
\item
  ces opérateurs sont également disponibles sous la forme de méthodes.
\end{itemize}

    \hypertarget{union}{%
\subparagraph{Union}\label{union}}

    \begin{Verbatim}[commandchars=\\\{\},frame=single,framerule=0.3mm,rulecolor=\color{cellframecolor}]
{\color{incolor}In [{\color{incolor}19}]:} \PY{n}{A2} \PY{o}{|} \PY{n}{A3}
\end{Verbatim}


\begin{Verbatim}[commandchars=\\\{\},frame=single,framerule=0.3mm,rulecolor=\color{cellframecolor}]
{\color{outcolor}Out[{\color{outcolor}19}]:} \{0, 2, 3, 4, 6\}
\end{Verbatim}
            
    \hypertarget{intersection}{%
\subparagraph{Intersection}\label{intersection}}

    \begin{Verbatim}[commandchars=\\\{\},frame=single,framerule=0.3mm,rulecolor=\color{cellframecolor}]
{\color{incolor}In [{\color{incolor}20}]:} \PY{n}{A2} \PY{o}{\PYZam{}} \PY{n}{A3}
\end{Verbatim}


\begin{Verbatim}[commandchars=\\\{\},frame=single,framerule=0.3mm,rulecolor=\color{cellframecolor}]
{\color{outcolor}Out[{\color{outcolor}20}]:} \{0, 6\}
\end{Verbatim}
            
    \hypertarget{diffuxe9rence}{%
\subparagraph{Différence}\label{diffuxe9rence}}

    \begin{Verbatim}[commandchars=\\\{\},frame=single,framerule=0.3mm,rulecolor=\color{cellframecolor}]
{\color{incolor}In [{\color{incolor}21}]:} \PY{n}{A2} \PY{o}{\PYZhy{}} \PY{n}{A3}
\end{Verbatim}


\begin{Verbatim}[commandchars=\\\{\},frame=single,framerule=0.3mm,rulecolor=\color{cellframecolor}]
{\color{outcolor}Out[{\color{outcolor}21}]:} \{2, 4\}
\end{Verbatim}
            
    \begin{Verbatim}[commandchars=\\\{\},frame=single,framerule=0.3mm,rulecolor=\color{cellframecolor}]
{\color{incolor}In [{\color{incolor}22}]:} \PY{n}{A3} \PY{o}{\PYZhy{}} \PY{n}{A2}
\end{Verbatim}


\begin{Verbatim}[commandchars=\\\{\},frame=single,framerule=0.3mm,rulecolor=\color{cellframecolor}]
{\color{outcolor}Out[{\color{outcolor}22}]:} \{3\}
\end{Verbatim}
            
    \hypertarget{diffuxe9rence-symuxe9trique}{%
\subparagraph{Différence symétrique}\label{diffuxe9rence-symuxe9trique}}

    On rappelle que \(A \Delta B = (A - B) \cup (B - A)\)

    \begin{Verbatim}[commandchars=\\\{\},frame=single,framerule=0.3mm,rulecolor=\color{cellframecolor}]
{\color{incolor}In [{\color{incolor}23}]:} \PY{n}{A2} \PY{o}{\PYZca{}} \PY{n}{A3}
\end{Verbatim}


\begin{Verbatim}[commandchars=\\\{\},frame=single,framerule=0.3mm,rulecolor=\color{cellframecolor}]
{\color{outcolor}Out[{\color{outcolor}23}]:} \{2, 3, 4\}
\end{Verbatim}
            
    \hypertarget{comparaisons}{%
\subsubsection{Comparaisons}\label{comparaisons}}

    Ici encore on se donne deux ensembles~:

    \begin{Verbatim}[commandchars=\\\{\},frame=single,framerule=0.3mm,rulecolor=\color{cellframecolor}]
{\color{incolor}In [{\color{incolor}24}]:} \PY{n}{superset} \PY{o}{=} \PY{p}{\PYZob{}}\PY{l+m+mi}{0}\PY{p}{,} \PY{l+m+mi}{1}\PY{p}{,} \PY{l+m+mi}{2}\PY{p}{,} \PY{l+m+mi}{3}\PY{p}{\PYZcb{}}
         \PY{n+nb}{print}\PY{p}{(}\PY{l+s+s1}{\PYZsq{}}\PY{l+s+s1}{superset}\PY{l+s+s1}{\PYZsq{}}\PY{p}{,} \PY{n}{superset}\PY{p}{)}
         \PY{n}{subset} \PY{o}{=}  \PY{p}{\PYZob{}}\PY{l+m+mi}{1}\PY{p}{,} \PY{l+m+mi}{3}\PY{p}{\PYZcb{}}
         \PY{n+nb}{print}\PY{p}{(}\PY{l+s+s1}{\PYZsq{}}\PY{l+s+s1}{subset}\PY{l+s+s1}{\PYZsq{}}\PY{p}{,} \PY{n}{subset}\PY{p}{)}
\end{Verbatim}


    \begin{Verbatim}[commandchars=\\\{\},frame=single,framerule=0.3mm,rulecolor=\color{cellframecolor}]
superset \{0, 1, 2, 3\}
subset \{1, 3\}
\end{Verbatim}

    \hypertarget{uxe9galituxe9}{%
\subparagraph{Égalité}\label{uxe9galituxe9}}

    \begin{Verbatim}[commandchars=\\\{\},frame=single,framerule=0.3mm,rulecolor=\color{cellframecolor}]
{\color{incolor}In [{\color{incolor}25}]:} \PY{n}{heteroclite} \PY{o}{==} \PY{n}{heteroclite2}
\end{Verbatim}


\begin{Verbatim}[commandchars=\\\{\},frame=single,framerule=0.3mm,rulecolor=\color{cellframecolor}]
{\color{outcolor}Out[{\color{outcolor}25}]:} True
\end{Verbatim}
            
    \hypertarget{inclusion}{%
\subparagraph{Inclusion}\label{inclusion}}

    \begin{Verbatim}[commandchars=\\\{\},frame=single,framerule=0.3mm,rulecolor=\color{cellframecolor}]
{\color{incolor}In [{\color{incolor}26}]:} \PY{n}{subset} \PY{o}{\PYZlt{}}\PY{o}{=} \PY{n}{superset}
\end{Verbatim}


\begin{Verbatim}[commandchars=\\\{\},frame=single,framerule=0.3mm,rulecolor=\color{cellframecolor}]
{\color{outcolor}Out[{\color{outcolor}26}]:} True
\end{Verbatim}
            
    \begin{Verbatim}[commandchars=\\\{\},frame=single,framerule=0.3mm,rulecolor=\color{cellframecolor}]
{\color{incolor}In [{\color{incolor}27}]:} \PY{n}{subset} \PY{o}{\PYZlt{}} \PY{n}{superset}
\end{Verbatim}


\begin{Verbatim}[commandchars=\\\{\},frame=single,framerule=0.3mm,rulecolor=\color{cellframecolor}]
{\color{outcolor}Out[{\color{outcolor}27}]:} True
\end{Verbatim}
            
    \begin{Verbatim}[commandchars=\\\{\},frame=single,framerule=0.3mm,rulecolor=\color{cellframecolor}]
{\color{incolor}In [{\color{incolor}28}]:} \PY{n}{heteroclite} \PY{o}{\PYZlt{}} \PY{n}{heteroclite2}
\end{Verbatim}


\begin{Verbatim}[commandchars=\\\{\},frame=single,framerule=0.3mm,rulecolor=\color{cellframecolor}]
{\color{outcolor}Out[{\color{outcolor}28}]:} False
\end{Verbatim}
            
    \hypertarget{ensembles-disjoints}{%
\subparagraph{Ensembles disjoints}\label{ensembles-disjoints}}

    \begin{Verbatim}[commandchars=\\\{\},frame=single,framerule=0.3mm,rulecolor=\color{cellframecolor}]
{\color{incolor}In [{\color{incolor}29}]:} \PY{n}{heteroclite}\PY{o}{.}\PY{n}{isdisjoint}\PY{p}{(}\PY{n}{A3}\PY{p}{)}
\end{Verbatim}


\begin{Verbatim}[commandchars=\\\{\},frame=single,framerule=0.3mm,rulecolor=\color{cellframecolor}]
{\color{outcolor}Out[{\color{outcolor}29}]:} True
\end{Verbatim}
            
    \hypertarget{pour-en-savoir-plus}{%
\subsubsection{Pour en savoir plus}\label{pour-en-savoir-plus}}

    Reportez vous à
\href{https://docs.python.org/3/library/stdtypes.html\#set-types-set-frozenset}{la
section sur les ensembles} dans la documentation Python.


    % Add a bibliography block to the postdoc
    
    
    

    
    
    
    

    

    \hypertarget{bonnes-pratiques-de-pruxe9sentation-de-code}{%
\section{Bonnes pratiques de présentation de
code}\label{bonnes-pratiques-de-pruxe9sentation-de-code}}

    \hypertarget{compluxe9ment---niveau-basique}{%
\subsection{Complément - niveau
basique}\label{compluxe9ment---niveau-basique}}

    \hypertarget{la-pep-008}{%
\subsubsection{La PEP-008}\label{la-pep-008}}

    On trouve \href{http://legacy.python.org/dev/peps/pep-0008/}{dans la
PEP-008 (en anglais)} les conventions de codage qui s'appliquent à toute
la librairie standard, et qui sont certainement un bon point de départ
pour vous aider à trouver le style de présentation qui vous convient.

Nous vous recommandons en particulier les sections sur

\begin{itemize}
\tightlist
\item
  \href{http://legacy.python.org/dev/peps/pep-0008/\#code-lay-out}{l'indentation}
\item
  \href{http://legacy.python.org/dev/peps/pep-0008/\#whitespace-in-expressions-and-statements}{les
  espaces}
\item
  \href{http://legacy.python.org/dev/peps/pep-0008/\#comments}{les
  commentaires}
\end{itemize}

    \hypertarget{un-peu-de-lecture-le-module-pprint}{%
\subsubsection{\texorpdfstring{Un peu de lecture : le module
\texttt{pprint}}{Un peu de lecture : le module pprint}}\label{un-peu-de-lecture-le-module-pprint}}

    Voici par exemple le code du module \texttt{pprint} (comme PrettyPrint)
de la librairie standard qui permet d'imprimer des données.

La fonction du module - le pretty printing - est évidemment accessoire
ici, mais vous pouvez y voir illustré

\begin{itemize}
\tightlist
\item
  le \emph{docstring} pour le module: les lignes de 11 à 35,
\item
  les indentations, comme nous l'avons déjà mentionné sont à 4 espaces,
  et sans tabulation,
\item
  l'utilisation des espaces, notamment autour des affectations et
  opérateurs, des définitions de fonction, des appels de
  fonctions\ldots{}
\item
  les lignes qui restent dans une largeur ``raisonnable'' (79
  caractères)
\item
  vous pouvez regarder notamment la façon de couper les lignes pour
  respecter cette limite en largeur.
\end{itemize}

    \begin{Verbatim}[commandchars=\\\{\},frame=single,framerule=0.3mm,rulecolor=\color{cellframecolor}]
{\color{incolor}In [{\color{incolor}1}]:} \PY{k+kn}{from} \PY{n+nn}{modtools} \PY{k}{import} \PY{n}{show\PYZus{}module\PYZus{}html}
        \PY{k+kn}{import} \PY{n+nn}{pprint}
        \PY{n}{show\PYZus{}module\PYZus{}html}\PY{p}{(}\PY{n}{pprint}\PY{p}{)}
\end{Verbatim}


\begin{Verbatim}[commandchars=\\\{\},frame=single,framerule=0.3mm,rulecolor=\color{cellframecolor}]
{\color{outcolor}Out[{\color{outcolor}1}]:} <IPython.core.display.HTML object>
\end{Verbatim}
            
    \hypertarget{espaces}{%
\subsubsection{Espaces}\label{espaces}}

    Comme vous pouvez le voir dans \texttt{pprint.py}, les règles
principales concernant les espaces sont les suivantes.

    \begin{itemize}
\item
  S'agissant des \textbf{affectations} et \textbf{opérateurs}, on fera

  \texttt{x\ =\ y\ +\ z}

  Et non pas

  \sout{\texttt{x=y+z}}

  Ni

  \sout{\texttt{x\ =\ y+z}}

  Ni encore

  \sout{\texttt{x=y\ +\ z}}
\end{itemize}

L'idée étant d'aérer de manière homogène pour faciliter la lecture.

    \begin{itemize}
\item
  On \textbf{déclare une fonction} comme ceci

  \texttt{def\ foo(x,\ y,\ z):}

  Et non pas comme ceci (un espace en trop avant la parenthèse ouvrante)

  \sout{\texttt{def\ foo\ (x,\ y,\ z):}}

  Ni surtout comme ceci (pas d'espace entre les paramètres)

  \sout{\texttt{def\ foo\ (x,y,z):}}
\end{itemize}

    \begin{itemize}
\item
  La même règle s'applique naturellement aux \textbf{appels de
  fonction}:

  \texttt{foo(x,\ y,\ z)}

  et non pas

  \sout{\texttt{foo\ (x,y,z)}}

  ni

  \sout{\texttt{def\ foo\ (x,\ y,\ z):}}
\end{itemize}

    Il est important de noter qu'il s'agit ici de \textbf{règles d'usage} et
non pas de règles syntaxiques; tous les exemples barrés ci-dessus sont
en fait \textbf{syntaxiquement corrects}, l'interpréteur les accepterait
sans souci; mais ces règles sont \textbf{très largement adoptées}, et
obligatoires pour intégrer du code dans la librairie standard.

    \hypertarget{coupures-de-ligne}{%
\subsubsection{Coupures de ligne}\label{coupures-de-ligne}}

    Nous allons à présent zoomer dans ce module pour voir quelques exemples
de coupure de ligne. Par contraste avec ce qui précède, il s'agit cette
fois surtout de \textbf{règles syntaxiques}, qui peuvent rendre un code
non valide si elles ne sont pas suivies.

    \hypertarget{coupure-de-ligne-sans-backslash}{%
\subparagraph{\texorpdfstring{Coupure de ligne sans \emph{backslash}
(\textbackslash{})}{Coupure de ligne sans backslash (\textbackslash{})}}\label{coupure-de-ligne-sans-backslash}}

    \begin{Verbatim}[commandchars=\\\{\},frame=single,framerule=0.3mm,rulecolor=\color{cellframecolor}]
{\color{incolor}In [{\color{incolor}2}]:} \PY{n}{show\PYZus{}module\PYZus{}html}\PY{p}{(}\PY{n}{pprint}\PY{p}{,} 
                         \PY{n}{beg}\PY{o}{=}\PY{l+s+s2}{\PYZdq{}}\PY{l+s+s2}{def pprint}\PY{l+s+s2}{\PYZdq{}}\PY{p}{,}
                         \PY{n}{end}\PY{o}{=}\PY{l+s+s2}{\PYZdq{}}\PY{l+s+s2}{def pformat}\PY{l+s+s2}{\PYZdq{}}\PY{p}{)}
\end{Verbatim}


\begin{Verbatim}[commandchars=\\\{\},frame=single,framerule=0.3mm,rulecolor=\color{cellframecolor}]
{\color{outcolor}Out[{\color{outcolor}2}]:} <IPython.core.display.HTML object>
\end{Verbatim}
            
    La fonction \texttt{pprint} (ligne \textasciitilde{}47) est une
commodité (qui crée une instance de \texttt{PrettyPrinter}, sur lequel
on envoie la méthode \texttt{pprint}).

Vous voyez ici qu'il n'est \textbf{pas nécessaire} d'insérer un
\emph{backslash} (\texttt{\textbackslash{}}) à la fin des lignes 50 et
51, car il y a une parenthèse ouvrante qui n'est pas fermée à ce stade.

De manière générale, lorsqu'une parenthèse ouvrante \texttt{(} - idem
avec les crochets \texttt{{[}} et accolades \texttt{\{} - n'est pas
fermée sur la même ligne, l'interpréteur suppose qu'elle sera fermée
plus loin et n'impose pas de \emph{backslash}.

    Ainsi par exemple on peut écrire sans \emph{backslash}:

\begin{Shaded}
\begin{Highlighting}[frame=lines,framerule=0.6mm,rulecolor=\color{asisframecolor}]
\NormalTok{valeurs }\OperatorTok{=}\NormalTok{ [ }
   \DecValTok{1}\NormalTok{,}
   \DecValTok{2}\NormalTok{,}
   \DecValTok{3}\NormalTok{,}
   \DecValTok{5}\NormalTok{,}
   \DecValTok{7}\NormalTok{,}
\NormalTok{]}
\end{Highlighting}
\end{Shaded}

Ou encore

\begin{Shaded}
\begin{Highlighting}[frame=lines,framerule=0.6mm,rulecolor=\color{asisframecolor}]
\NormalTok{x }\OperatorTok{=}\NormalTok{ un_nom_de_fonction_tres_tres_long(}
\NormalTok{       argument1, argument2,}
\NormalTok{       argument3, argument4,}
\NormalTok{    )}
\end{Highlighting}
\end{Shaded}

    À titre de rappel, signalons aussi les chaînes de caractères à base de
\texttt{"""} ou
\texttt{\textquotesingle{}\textquotesingle{}\textquotesingle{}} qui
permettent elles aussi d'utiliser plusieures lignes consécutives sans
\emph{backslash}, comme~:

\begin{Shaded}
\begin{Highlighting}[frame=lines,framerule=0.6mm,rulecolor=\color{asisframecolor}]
\NormalTok{texte }\OperatorTok{=} \StringTok{"""Les sanglots longs}
\StringTok{Des violons}
\StringTok{De l'automne"""}
\end{Highlighting}
\end{Shaded}

    \hypertarget{coupure-de-ligne-avec-backslash}{%
\subparagraph{\texorpdfstring{Coupure de ligne avec \emph{backslash}
(\textbackslash{})}{Coupure de ligne avec backslash (\textbackslash{})}}\label{coupure-de-ligne-avec-backslash}}

    Par contre il est des cas où le backslash est nécessaire:

    \begin{Verbatim}[commandchars=\\\{\},frame=single,framerule=0.3mm,rulecolor=\color{cellframecolor}]
{\color{incolor}In [{\color{incolor}3}]:} \PY{n}{show\PYZus{}module\PYZus{}html}\PY{p}{(}\PY{n}{pprint}\PY{p}{,} 
                         \PY{n}{beg}\PY{o}{=}\PY{l+s+s2}{\PYZdq{}}\PY{l+s+s2}{components), readable, recursive}\PY{l+s+s2}{\PYZdq{}}\PY{p}{,} 
                         \PY{n}{end}\PY{o}{=}\PY{l+s+s2}{\PYZdq{}}\PY{l+s+s2}{elif len(object) }\PY{l+s+s2}{\PYZdq{}}\PY{p}{,} 
                         \PY{n}{lineno\PYZus{}width}\PY{o}{=}\PY{l+m+mi}{3}\PY{p}{)}
\end{Verbatim}


\begin{Verbatim}[commandchars=\\\{\},frame=single,framerule=0.3mm,rulecolor=\color{cellframecolor}]
{\color{outcolor}Out[{\color{outcolor}3}]:} <IPython.core.display.HTML object>
\end{Verbatim}
            
    Dans ce fragment au contraire, vous voyez en ligne 521 qu'\textbf{il a
fallu cette fois} insérer un \emph{backslash} \texttt{\textbackslash{}}
comme caractère de continuation pour que l'instruction puisse se
poursuivre en ligne 522.

    \hypertarget{coupures-de-lignes---uxe9pilogue}{%
\subparagraph{Coupures de lignes -
épilogue}\label{coupures-de-lignes---uxe9pilogue}}

    Dans tous le cas où une instruction est répartie sur plusieurs lignes,
c'est naturellement l'indentation de \textbf{la première ligne} qui est
significative pour savoir à quel bloc rattacher cette instruction.

    Notez bien enfin qu'on peut toujours mettre un \emph{backslash} même
lorsque ce n'est pas nécessaire, mais on évite cette pratique en règle
générale car les \emph{backslash} nuisent à la lisibilité.

    \hypertarget{compluxe9ment---niveau-intermuxe9diaire}{%
\subsection{Complément - niveau
intermédiaire}\label{compluxe9ment---niveau-intermuxe9diaire}}

    \hypertarget{outils-liuxe9s-uxe0-pep008}{%
\subsubsection{Outils liés à PEP008}\label{outils-liuxe9s-uxe0-pep008}}

Il existe plusieurs outils liés à la PEP0008, pour vérifier si votre
code est conforme, ou même le modifier pour qu'il le devienne.

Ce qui nous donne un excellent prétexte pour parler un peu de
\url{https://pypi.python.org}, qui est la plateforme qui distribue les
logiciels disponibles via l'outil \texttt{pip3}.

Je vous signale notamment:

\begin{itemize}
\tightlist
\item
  \href{https://pypi.python.org/pypi/pep8/}{l'outil \texttt{pep8}} pour
  vérifier, et
\item
  \href{https://pypi.python.org/pypi/autopep8/}{l'outil
  \texttt{autopep8}} pour modifier automatiquement votre code et le
  rendre conforme.
\end{itemize}

    \hypertarget{les-deux-points}{%
\subsubsection{Les deux-points `:'}\label{les-deux-points}}

    Dans un autre registre entièrement, vous pouvez
\href{https://docs.python.org/3/faq/design.html\#why-are-colons-required-for-the-if-while-def-class-statements}{vous
reporter à ce lien} si vous êtes intéressé par la question de savoir
pourquoi on a choisi un délimiteur (le caractère deux-points \texttt{:})
pour terminer les instructions comme \texttt{if}, \texttt{for} et
\texttt{def}.


    % Add a bibliography block to the postdoc
    
    
    
